\section{Collaborating IoT Robots}

Robots will need to collaborate to fulfill tasks that they cannot do alone at some point.
To collaborate, they need to communicate to each other their "skills" (what they can or cannot do, example : "I can move on an plane floor, but not stairs").
Once all the skills have been collected, compared with a mission statement, a mission plan can be generated.
Once the plan is generated, the controllers can execute this plan via standard communication interfaces.

In the case of robot modularity, where every robot parts can be considered as a standalone system collaborating with others to make a whole robot, this kind of system would enable hardware constructor to focus on the hardware itself.
With a proper description of the hardware skills, a robot could use a new module with no need to modify the robot's software.
This would be even more useful if robots are able to remove/add/replace parts autonomously.

In the case of a task that is too hard for one robot, the robot could autonomously ask the help of other robots or external sensors to reach it's goals.

The computing power needed for this kind of system is likely to be too high for being embedded, considering the complexity of missions the robots will be needed to accomplish.
There is a need to distribute the computing tasks, or (at least partly) execute it in the cloud, taking advantage of the omnipresence of wireless networking.

\subsection{A concrete use-case}

\todoEB{reference + quickly describe your mission at the Robotics French Cup.}

These ideas can be applied in the field of modular robotics, and for collaboration between robots.

Eurobot is an international competition for students and young hobbyistes, where the goal is to design robots that are able to accomplish several actions (generally 5).
For each match, 2 teams of 2 robots have 100 seconds to score, performing the actions on the same table.

\subsection{Related work}

\todoEB{short paragraph referring papers (de la conf que tu avais repéré), software (Europa, ROS...) etc. that you have found in the domain}

% points de départ : 
%% https://github.com/nasa/europa/wiki/Europa-Publications
%% http://kcl-planning.github.io/ROSPlan/documentation/

 - IoT robots (contrained microcontrollers) \cite{EWSN-IoT-Robot} have been ported to ROS2 with the help of RIOT and NDN \cite{ACM-ICN-IoT-Robotics}.
 - There is some work to design an architecture for planning with ROS \cite{ROSPlan}.
 - There is also some work to distribute planning system ovel several subsystems \cite{NASA-europa-p2p}.
 - A konwledge sharing database for robot have been described in \cite{Roboearth}

\todoEB{a short explanation what is the contribution of this paper compared to such prior work}


