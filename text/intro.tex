\section{Collaborating IoT Robots}

Robots will need to collaborate to fulfill tasks that they cannot do alone at some point.
To collaborate, they need to communicate to each other their "skills" (what they can or cannot do, example : "I can move on an plane floor, but not stairs").
Once all the skills have been collected, compared with a mission statement, a mission plan can be generated.
Once the plan is generated, the controllers can execute this plan via standard communication interfaces.

In the case of robot modularity, where every robot parts can be considered as a standalone system collaborating with others to make a whole robot, this kind of system would enable hardware constructor to focus on the hardware itself.
With a proper description of the hardware skills, a robot could use a new module with no need to modify the robot's software.
This would be even more useful if robots are able to remove/add/replace parts autonomously.

In the case of a task that is too hard for one robot, the robot could autonomously ask the help of other robots or external sensors to reach it's goals.

The computing power needed for this kind of system is likely to be too high for being embedded, considering the complexity of missions the robots will be needed to accomplish.
There is a need to distribute the computing tasks, or (at least partly) execute it in the cloud, taking advantage of the omnipresence of wireless networking.

\subsection{A concrete use-case}

\todoEB{reference + quickly describe your mission at the Robotics French Cup.}

These ideas can be applied in the field of modular robotics, and for collaboration between robots.

Eurobot is an international competition for students and young hobbyistes, where the goal is to design robots that are able to accomplish several actions (generally 5).
For each match, 2 teams of 2 robots have 100 seconds to score, performing the actions on the same table.

% bon, je jette des idées en vrac, j'arrive pas à rédiger...
 - Our team will be composed of two-wheeled robots
 - The robot is designed on a modular way.
 - Each module may have a control board that is linked to the central CAN bus (widely used in industry)
 - The main module is the Rolling Base, which is controlled by a Teensy 3.2
 - The robot may embed a raspberry pi to handle some heavy computing algorithm (A*, Planner ?)
 - Some short range radio communication may be used (Wifi ? Bluetooth ? 802.15.4 ?)
 - For the competition, each year, the modules will need to be changed.
 - The idea would be that if we design the mecanics, and write the "skills" of the module, the robot could integrate it with no further programming !!

 - We will also have 3 external sensors.
 - The sensors will be cameras that will recognize some 2D pattern put on the robot to compute it's position. (the team's robots, and adversary robots).
 - The 3 sensors will communicate with each other to triangulate the position
 - and even communicate with the robots to have a better evaluation (kalman/particle filter ?)
 - We could add other sensors (other cameras) to track the objects on the table (machine learning ?)
 
 - There is also some space where we could put more computing power if needed (simulating the cloud, because communication with the external world is not allowed)

\subsection{Related work}

\todoEB{short paragraph referring papers (de la conf que tu avais repéré), software (Europa, ROS...) etc. that you have found in the domain}

\todoEB{a short explanation what is the contribution of this paper compared to such prior work}
