\section{Prototype Implementation}
\todoEB{Describe the actual implementations/parts that you are putting together for the Robotics French Cup use case.}

% bon, je jette des idées en vrac, j'arrive pas à rédiger...
 - Our team will be composed of two-wheeled robots
 - The robot is designed on a modular way.
 - Each module may have a control board that is linked to the central CAN bus (widely used in industry)
 - The main module is the Rolling Base, which is controlled by a Teensy 3.2
 - The robot may embed a raspberry pi to handle some heavy computing algorithm (A*, Planner ?)
 - Some short range radio communication may be used (Wifi ? Bluetooth ? 802.15.4 ?)
 - For the competition, each year, the modules will need to be changed.
 - The idea would be that if we design the mecanics, and write the "skills" of the module, the robot could integrate it with no further programming !!

 - We will also have 3 external sensors.
 - The sensors will be cameras that will recognize some 2D pattern put on the robot to compute it's position. (the team's robots, and adversary robots).
 - The 3 sensors will communicate with each other to triangulate the position
 - and even communicate with the robots to have a better evaluation (kalman/particle filter ?)
 - We could add other sensors (other cameras) to track the objects on the table (machine learning ?)
 
 - There is also some space where we could put more computing power if needed (simulating the cloud, because communication with the external world is not allowed)
