\documentclass[conference]{IEEEtran}
\usepackage{blindtext, graphicx}

% correct bad hyphenation here
\hyphenation{op-tical net-works semi-conduc-tor}


\begin{document}
\title{Software and Networking infrastructure \\for IoT and Robotics collaboration}

\author{
\IEEEauthorblockN{Loic Dauphin}
\IEEEauthorblockA{
Inria, Universit\'e Paris-Saclay\\
loic.dauphin@inria.fr
}
\and
\IEEEauthorblockN{Homer Simpson}
\IEEEauthorblockA{
Springfield, USA\\
Email: homer@thesimpsons.com
}
\and
\IEEEauthorblockN{James Kirk\\ and Montgomery Scott}
\IEEEauthorblockA{Starfleet Academy}
}

\maketitle

\begin{abstract}
%\blindtext
\end{abstract}

\begin{IEEEkeywords}
IEEEtran, journal, \LaTeX, paper, template.
\end{IEEEkeywords}

\IEEEpeerreviewmaketitle

\section{Robotics/IoT collaboration}

Robots will need to collaborate to fulfill tasks that they cannot do alone at some point.
To collaborate, they need to communicate to each other their "skills" (what they can or cannot do, example : "I can move on an plane floor, but not stairs").
Once all the skills have been collected, compared with a mission statement, a mission plan can be generated.
Once the plan is generated, the controllers can execute this plan via standard communication interfaces.

In the case of robot modularity, where every robot parts can be considered as a standalone system collaborating with others to make a whole robot, this kind of system would enable hardware constructor to focus on the hardware itself.
With a proper description of the hardware skills, a robot could use a new module with no need to modify the robot's software.
This would be even mode useful if robots are able to remove/add/replace parts autonomousely.

In the case of a task that is too hard for one robot, the robot could ask the help of other robots or external sensors to reach it's goals.

\section{A common communication protocol}

\subsection{Needed patterns}
\subsubsection{Discovery}
\subsubsection{Pub/Sub}
\subsubsection{Req/Res}

\subsection{The interface of each entities}
\subsubsection{Control interface (sense+act)}
\subsubsection{Skills/Constraints (Plan)}

\section{A "collaboration" planner}

\subsection{Expressing the mission of a robot}
\subsection{Need to discover and get information for planning from control nodes}
\subsection{Describing the environment + control/planning loop}

\begin{thebibliography}{1}

\bibitem{IEEEhowto:kopka}
H.~Kopka and P.~W. Daly, \emph{A Guide to \LaTeX}, 3rd~ed.\hskip 1em plus
  0.5em minus 0.4em\relax Harlow, England: Addison-Wesley, 1999.

\end{thebibliography}

\end{document}
